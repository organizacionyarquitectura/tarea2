% Tipo de documento 
\documentclass{article}

% Idioma
\usepackage[spanish]{babel}

% Márgenes amplios
\usepackage[margin = 1.5cm]{geometry}

% Algunos comandos pra trabajar con ecuaciones (alinear)
\usepackage{amsmath}

\begin{document}
    % Título
    \title{Organización y Arquitectura de Computadoras \\
    \large 2912-2 \\
    \large Sistemas numéricos}

    \date{Fecha de entrega: 15 de febrero del 2019}

    \author{Edgar Quiroz Castañeda}  

    \maketitle
    
    % Ejercicios
    \section{Conversiones}
    Realiza las siguientes transformaciones
    \begin{enumerate}
        \item {
            $101110011_2 \rightarrow_{16} \rightarrow_{10}$
            \begin{itemize}
                \item {
                    $\rightarrow_{16}$ \\
                    Agrupando por grupos de cuatro dígitos, tenemos que 
                    \begin{align*}
                        0011_2 &= 3_{16} \\
                        0111_2 &= 7_{16} \\
                        0001_2 &= 1_{16}
                    \end{align*}
                    Por lo que $101110011_2 \rightarrow_{16} 173_{16}$
                }
                \item {
                    $\rightarrow_{10}$\\
                    \begin{align*}
                        101110011_2 &= (2^8 + 2^6 + 2^5 + 2^4 + 2^1 + 2^0)_{10} \\
                                    &= (256+64+32+16+2+1)_{10} = 
                    \end{align*}

                }
            \end{itemize}
        }
        \item {
            $347_8 \rightarrow_2 \rightarrow_10$
        }
        \item {
            $45AF_16 \rightarrow_8 \rightarrow_10$
        }
    \end{enumerate}

    \section{Números negativos}
    Transforma los siguientes números a su representaciones en signo y magnitud,
    complemento a 1, complemento a 2 y exceso a 128.

    \begin{enumerate}
        \item {
            87
        }
        \item {
            -34
        }
        \item {
            21
        }
        \item {
            -53
        }
    \end{enumerate}

    \section{Números de punto flotante}
    \begin{enumerate}
        \item {
            Obtén los 32 bits a partir de número 12.358
        }
        \item {
            Obtén los 32 bits a partir de número 0.1592
        }
        \item {
            Encuentra el número decimal a partir de 
            01000010010011001101001010001001
        }
        \item {
            Encuentra el número decimal a partir de 0xbefbda51
        }
    \end{enumerate}
\end{document}
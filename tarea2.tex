% Tipo de documento 
\documentclass{article}

% Idioma
\usepackage[spanish]{babel}

% Márgenes amplios
\usepackage[margin = 1.5cm]{geometry}

% Algunos comandos para trabajar con ecuaciones (alinear)
\usepackage{amsmath}

% Colores (para las partes del número con punto flotante)
\usepackage{xcolor}

\begin{document}
    % Título
    \title{Organización y Arquitectura de Computadoras \\
    \large 2912-2 \\
    \large Sistemas numéricos}

    \date{Fecha de entrega: 15 de febrero del 2019}

    \author{Edgar Quiroz Castañeda}  

    \maketitle
    
    % Ejercicios
    \section{Conversiones}
    Realiza las siguientes transformaciones
    \begin{enumerate}
        \item {
            $101110011_2 \rightarrow_{16} \rightarrow_{10}$
            \begin{itemize}
                \item {
                    $\rightarrow_{16}$ \\
                    Agrupando por grupos de cuatro dígitos, tenemos que 
                    \begin{align*}
                        0011_2 &= 3_{16} \\
                        0111_2 &= 7_{16} \\
                        0001_2 &= 1_{16}
                    \end{align*}
                    Por lo que $101110011_2 \rightarrow_{16} 173_{16}$
                }
                \item {
                    $\rightarrow_{10}$
                    \begin{align*}
                        101110011_2 &= (2^8 + 2^6 + 2^5 + 2^4 + 2^1 + 2^0)_{10} \\
                                    &= (256+64+32+16+2+1)_{10} \\
                                    &= 371_{10}
                    \end{align*}

                }
            \end{itemize}
        }
        \item {
             $347_8 \rightarrow_2 \rightarrow_{10}$
            \begin{itemize}
                \item {
                    $\rightarrow_{2}$ \\
                    Llevando a cada dígito a su representación binaria de tres dígitos, 
                    tenemos que 
                    \begin{align*}
                        7_{8} &= 111_{2} \\
                        4_{8} &= 100_{2} \\
                       3_{8} &= 011_{2}
                     \end{align*}
                    Por lo que $347_8 \rightarrow_2 011100111_{2}$
                }
                \item {
                    $\rightarrow_{10}$
                    \begin{align*}
                        347_{8} &= (3 \times 8^2 + 4 \times 8^1 + 7 \times 8^0)_{10} \\
                                    &= (3 \times 64 + 4 \times 8 + 7)_{10} \\
                                    &= (192 + 32 + 7) \\
                                    &= 231
                    \end{align*}

                }
            \end{itemize}
        }
        \item {
            $45AF_{16} \rightarrow_8 \rightarrow_{10}$
            \begin{itemize}
                \item {
                    $\rightarrow_8$ \\
                    Como paso intermedio, se pasa el número primero a binario.\\
                    Entonces, llevando a cada dígito a su representación 
                    binaria de cuatro dígitos, tenemos que 
                    \begin{align*}
                        4 &= 0100 \\
                        5 &= 0101 \\
                        A &= 1010 \\
                        F &= 1111
                    \end{align*}
                    Por lo que $45AF_{16} = 0100010110101111_{2}$. \\
                    Luego, agrupando por grupos de tres dígitos,
                    \begin{align*}
                        111_{2} &= 7_{8} \\
                        101_{2} &= 5_{8} \\
                        110_{2} &= 6_{8} \\
                        010_{2} &= 2_{8} \\
                        100_{2} &= 4_{8}
                    \end{align*}
                    Por lo que $45AF_{16} = 0100010110101111_{2} = 42657_{8}$
                }
                \item {
                    $\rightarrow_{10}$
                    \begin{align*}
                        45AF_{16} &= (4 \times 16^3 + 5 \times 16^2
                        + 10 \times 16^1 + 15 \times 16^0)_{10} \\
                                    &= (4 \times 4096 + 5 \times 256 
                                    + 10 \times 16 + 15)_{10} \\
                                    &= (16384 + 1280 + 160 + 15)_{10} \\
                                    &= 17839    _{10}
                    \end{align*}
                }
            \end{itemize}
        }
    \end{enumerate}

    \section{Números negativos}
    Transforma los siguientes números a su representaciones en signo y magnitud,
    complemento a 1, complemento a 2 y exceso a 128.

    \begin{enumerate}
        \item {
            87 \\
            Primero, hay que obtener la representación binaria
            \begin{align*}
                87 &= 43 \times 2 + 1 \\
                43 &= 21 \times 2 + 1 \\
                21 &= 10 \times 2 + 1 \\
                10 &= 5 \times 2 + 0 \\
                5 &= 2 \times 2 + 1 \\
                2 &= 2 \times 1 + 0 \\
                1 &= 0 \times 0 + 1
            \end{align*}
            Por lo que $87_{10} = 1010111_{2}$.\\
            Las representaciones son 
            \begin{itemize}
                \item {
                    Signo y magnitud \\
                    Sólo hay que agregar un bit para el signo.
                    \[r = 01010111\]
                    }
                \item {
                    Complemento a 1 y 2\\
                    Como el número es positivo, entonces ya está
                    en su forma de complemento a 1 y 2.
                    \[r = 1010111\]
                    }
                    \item {
                    Exceso a 128
                    \[r_{10} = 87 + 128 = 215\]
                    Ahora hay que pasar eso a binario
                    \begin{align*}
                        215 &= 107 \times 2 + 1 \\
                        107 &= 53 \times 2 + 1 \\
                        53 &= 26 \times 2 + 1 \\
                        26 &= 13 \times 2 + 1 \\
                        13 &= 6 \times 2 + 1 \\
                        6 &= 3 \times 2 + 0 \\
                        3 &= 1 \times 2 + 1 \\
                        1 &= 0 \times 2 + 1
                    \end{align*}
                    Entonces
                    \[r = 11011111\]
                    }
            \end{itemize}
        }
        \item {
            -34 \\
            Primero hay que pasar el número sin signo a binario.
            \begin{align*}
                34 &= 17 \times 2 + 0 \\
                17 &= 8 \times 2 + 1 \\
                8 &= 4 \times 2 + 0 \\
                4 &= 2 \times 2 + 0 \\
                2 &= 2 \times 1 + 0 \\
                1 &= 0 \times 2 + 1
            \end{align*}
            Por lo que $34_{10} = 100010_{2}$
            Las representaciones son 
            \begin{itemize}
                \item {
                    Signo y magnitud \\
                    Sólo hay que agregar un bit para el signo.
                    \[r = 11010111\]
                }
                \item {
                    Complemento a 1\\
                    Como el número es negativo, hay invertir sus dígitos para 
                    obtener su complemento.
                    \[r = 0101000\]
                }
                \item {
                    Complemento a 2\\
                    Como el número es negativo, hay que encontrar su primer 1 
                    desde la derecha e invertir todos los dígitos desde ese 
                    punto para obtener su complemento. Como su primer dígito 
                    desde la derecha es 1, entonces esto es lo mismo que 
                    invertir todos sus dígitos.
                    \[r = 0101000\]
                }
                \item {
                    Exceso a 128
                    \[r_{10} = -34 + 128 = 94\]
                    Ahora hay que pasar eso a binario
                    \begin{align*}
                        94 &= 47 \times 2 + 0 \\
                        47 &= 23 \times 2 + 1 \\
                        23 &= 11 \times 2 + 1 \\
                        11 &= 5 \times 2 + 1 \\
                        5 &= 2 \times 2 + 1 \\
                        2 &= 1 \times 2 + 0 \\
                        1 &= 0 \times 2 + 1 
                    \end{align*}
                    Entonces
                    \[r = 1011110\]
                }
            \end{itemize}
        }
        \item {
            21 \\
            Primero, hay que obtener la representación binaria
            \begin{align*}
                21 &= 10 \times 2 + 1 \\
                10 &= 5 \times 2 + 0 \\
                5 &= 2 \times 2 + 1 \\
                2 &= 1 \times 2 + 0 \\
                1 &= 0 \times 2 + 1 
            \end{align*}
            Por lo que $21_{10} = 10101_{2}$.\\
            Las representaciones son 
            \begin{itemize}
                \item {
                    Signo y magnitud \\
                    Sólo hay que agregar un bit para el signo.
                    \[r = 010101\]
                    }
                \item {
                    Complemento a 1 y 2\\
                    Como el número es positivo, entonces ya está
                    en su forma de complemento a 1 y 2.
                    \[r = 10101\]
                    }
                    \item {
                    Exceso a 128
                    \[r_{10} = 21 + 128 = 139\]
                    Ahora hay que pasar eso a binario
                    \begin{align*}
                        139 &= 69 \times 2 + 1 \\
                        69 &= 34 \times 2 + 1 \\
                        34 &= 19 \times 2 + 0 \\
                        19 &= 9 \times 2 + 1 \\
                        9 &= 4 \times 2 + 1 \\
                        4 &= 2 \times 2 + 0 \\
                        2 &= 1 \times 2 + 0 \\
                        1 &= 0 \times 2 + 1
                    \end{align*}
                    Entonces
                    \[r = 10011011\]
                    }
            \end{itemize}
        }
        \item {
            -53 \\
            Primero hay que pasar el número sin signo a binario.
            \begin{align*}
                53 &= 26 \times 2 + 1 \\
                26 &= 13 \times 2 + 0 \\
                13 &= 6 \times 2 + 1 \\
                6 &= 3 \times 2 + 0 \\
                3 &= 1 \times 2 + 1 \\
                1 &= 0 \times 2 + 1
            \end{align*}
            Por lo que $53_{10} = 110101_{2}$. \\
            Las representaciones son 
            \begin{itemize}
                \item {
                    Signo y magnitud \\
                    Sólo hay que agregar un bit para el signo.
                    \[r = 1110101\]
                }
                \item {
                    Complemento a 1\\
                    Como el número es negativo, hay invertir sus dígitos para 
                    obtener su complemento.
                    \[r = 001010\]
                }
                \item {
                    Complemento a 2\\
                    Como el número es negativo, hay que encontrar su primer 1 
                    desde la derecha e invertir todos los dígitos desde ese 
                    punto para obtener su complemento. Como su primer dígito 
                    desde la derecha es 1, entonces esto es lo mismo que 
                    invertir todos sus dígitos.
                    \[r = 001010\]
                }
                \item {
                    Exceso a 128
                    \[r_{10} = -53 + 128 = 75\]
                    Ahora hay que pasar eso a binario
                    \begin{align*}
                        75 &= 37 \times 2 + 1 \\
                        37 &= 18 \times 2 + 1 \\
                        18 &= 9 \times 2 + 0 \\
                        9 &= 4 \times 2 + 1 \\
                        4 &= 2 \times 2 + 0 \\
                        2 &= 1 \times 2 + 0 \\
                        1 &= 0 \times 2 + 1 
                    \end{align*}
                    Entonces
                    \[r = 1001011\]
                }
            \end{itemize}
        }
    \end{enumerate}

    \section{Números de punto flotante}
    \begin{enumerate}
        \item {
            Obtén los 32 bits a partir de número 12.358 \\
            Primero hay que pasar la parte entera a binario
            \begin{align*}
                12 &= 6 \times 2 + 0 \\
                6 &= 3 \times 2 + 0 \\
                3 &= 1 \times 2 + 1 \\
                1 &= 0 \times 2 + 1
            \end{align*}
            Por lo que $12_{10} = 1100_{2}$.\\
            Ahora hay que pasar la parte decimal a binario. Por simplicidad, hay
            que aproximarlo sólo a 5 decimales.
            \begin{align*}
                0.358 \times 2 &= 0.716 \\
                0.716 \times 2 &= 1.432 \\
                0.432 \times 2 &= 0.864 \\
                0.864 \times 2 &= 1.728 \\
                0.728 \times 2 &= 1.456
            \end{align*}
            Por lo que $12.458_{10} \approx 1100.01011_{2} 
            = 1.10001011_{2} \times 2 ^ 3$.\\
            Luego, hay que encontrar el valor en binario del exponente con exceso
            a 127.
            \begin{align*}
                3 + 127 &= 130 \\
                130 &= 65 \times 2 + 0 \\
                65 &= 32 \times 2 + 1 \\
                32 &= 16 \times 2 + 0 \\
                16 &= 8 \times 2 + 0 \\
                8 &= 4 \times 2 + 0 \\
                4 &= 2 \times 2 + 0 \\
                2 &= 1 \times 2 + 0 \\
                1 &= 0 \times 2 + 1
            \end{align*}
            Por lo que $130_{10} = 10000010_{2}$ \\
            Y como es potivo, el bit del signo es 0. \\
            Por lo que la representación en 32 bits es
            \[r = {\color{red}0}{\color{green}10000010}
            {\color{blue}10001011000000000000000}\]
        }
        \item {
            Obtén los 32 bits a partir de número 0.1592 \\
            Hay que pasar la parte decimal a binario. Por simplicidad, hay
            que aproximarlo sólo a 5 decimales.
            \begin{align*}
                0.1592 \times 2 &= 0.3184 \\
                0.3184 \times 2 &= 0.6368 \\
                0.6368 \times 2 &= 1.2736 \\
                0.2736 \times 2 &= 0.5472 \\
                0.5472 \times 2 &= 1.0.944
            \end{align*}
            Por lo que $0.1592_{10} \approx 0.00101_{2} 
            = 1.01_{2} \times 2 ^ {-3}$.\\
            Luego, hay que encontrar el valor en binario del exponente con exceso
            a 127.
            \begin{align*}
                -3 + 127 &= 124 \\
                124 &= 62 \times 2 + 0 \\
                62 &= 31 \times 2 + 0 \\
                31 &= 15 \times 2 + 1 \\
                15 &= 7 \times 2 + 1 \\
                7 &= 3 \times 2 + 1 \\
                3 &= 1 \times 2 + 1 \\
                1 &= 0 \times 2 + 1
            \end{align*}
            Por lo que $124_{10} = 1111100_{2}$ \\
            Y como es positivo, el bit del signo es 0. \\
            Por lo que la representación en 32 bits es
            \[r = {\color{red}0}{\color{green}01111100}
            {\color{blue}01000000000000000000000}\]
        }
        \item {
            Encuentra el número decimal a partir de 
            01000010010011001101001010001001\\
            Como el primer dígito es 0, es positivo. Como la mantisa es 
            10000100, entonces el exponente es $10000100_{2} -127_{10} 
            = (132 - 127)_{10} = 5_{10}$.\\
            Por lo que el número es
            \[1.10011001101001010001001 \times 2 ^ 5 = 110011.001101001010001001\]
            Y esto en decimal es
            \begin{align*}
                 &= 
                2^5 + 2^4 + 2^1 + 2^0 + 2^{-3} + 2^{-4} + 2^{-6} + 2^{-9} 
                + 2^{-11} + 2^{-15} + 2^{-18} \\
                &= 32 + 16 + 2 + 1 + 0.125 + 0.625 + 0.015625 + 0.001953125 \\ 
                &+ 0.00048828125 + 0.000030517578125 + 0.000003814697265625 \\
                &= 51.205600738525390625
            \end{align*}
            
        }
        \item {
            Encuentra el número decimal a partir de 0xbefbda51.\\
            Hay que pasar el número a binario, agrupando por grupos de 4 dígitos.
            \begin{align*}
                b &= 1011 \\
                e &= 1110 \\
                f &= 1111 \\
                b &= 1011 \\
                d &= 1101 \\
                a &= 1010 \\
                5 &= 0101 \\
                1 &= 0001
            \end{align*}
            Por lo que 0xbefbda51 = 10111110111110111101101001010001. \\
            Como el primer dígito es 1, es negativo. Como la mantisa es 
            01111101, entonces el exponente es $01111101_{2} -127_{10} 
            = (125 - 127)_{10} = -2_{10}$. \\
            Por lo que el número es 
            \[-1.11110111101101001010001 \times 2 ^ {-2} 
            = -0.0111110111101101001010001\]
            Que en decimal es
            \begin{align*}
                &= -(2^{-2} +  2^{-3} +  2^{-4} +  2^{-5} +  2^{-6} 
                +  2^{-8} +  2^{-9} +  2^{-10} +  2^{-11} +  2^{-13} 
                +  2^{-14} +  2^{-16} +  2^{-19} +  2^{-21} +  2^{-25}) \\
                &= -(0.25 + 0.125 + 0.0625 + 0.03125 + 0.015625 + 0.00390625 
                + 0.001953125 + 0.0009765625 \\
                &+ 0.00048828125 + 0.0001220703125 
                + 0.00006103515625 + 0.0000152587890625 \\
                &+ 0.0000019073486328125 + 0.000000476837158203125 
                + 0.00000002980232238769531250) \\
                &= -0.4918999969959259033203125
            \end{align*}
        }
    \end{enumerate}
\end{document}